Me chamo Aléssio Miranda Júnior, atualmente professor EBTT a 10 anos no Cefet-MG campus Timóteo. Academicamente me graduei na UFV em 2006, conclui meu mestrado na UFPR em 2009, sendo egresso na primeira turma de Doutorado da instituição ainda em 2009. Em 2010 fiz a opção de assumir meu cargo junto ao Cefet-MG e acabei não conseguindo dar prosseguimento ao meu projeto de Doutoramento (na área de Linguística Computacional, com o tema “Uma Máquina de tradução automática LIBRAS-Português baseado em Regras sintáticas Superficiais”), a princípio pela distância, e pela sobrecarga que tive no campus. Não havia redução de carga horária, nem previsão de afastamento sendo que acabei me desligando. 

Durante os anos de 2011 a 2013 me dediquei ao campus, mas mantive contato com minha orientadora, sendo ativo no grupo remotamente e no fim de 2013 fiz um reingresso no programa da UFPR. Ainda sem licença no primeiro ano optei por viajar ao paraná de 15 em 15 dias, me desdobrando para fazer as disciplinas presenciais, semi presencialmente e no inicio de 2015 consegui uma e fiquei 12 meses em Curitiba. 

Nesta segunda tentativa percebi que eu estava desgastado com o tema, (que eu insistia desde 2009 e que apresentavam desafios interdisciplinares que eu tive dificuldades em superar principalmente o conhecimento linguístico de LIBRAS), após perceber que eu estava em uma reta descendente de produção, em um projeto que eu tinha perdido o entusiasmo e apresentou desafios que eu não consegui enfrentar escolhi junto com minha orientadora que era melhor eu me desligar, pois mudar de tema na situação que eu me encontrava não parecia a melhor escolha naquele momento.

Além do meu desinteresse com o tema e como minha (ex)orientadora também iniciou seu processo de aposentadoria, de 2016 a 2020, passei por um período de redescoberta de temas, buscando algum tipo de temática que realmente despertasse meu interesse a ponto de ingressar novamente em um processo de doutoramento. Mesmo com as dificuldades inerentes do meu departamento, tenho tentado conhecer novos temas, auxiliado e escrito artigos para conferencias a fim de me manter ativo e com a mente aberta.

Mas desde 2016, com a cabeça mais fria e descansada, comecei a me envolver gradativamente com uma temática que me intrigava desde a graduação, depois na UFPR e ainda no CEFET-MG. Em primeiro momento passei a estudar as metodologias de ensino programação e me envolver mais diretamente com o treinamento de equipes de maratona de programação. 
	
Desde então, me dediquei a testar nas minhas disciplinas ministradas no Cefet-MG, de forma \textit{Ad-Hoc}, metodologias e técnicas de ensino de computação usando princípios de competições Competitivas como a Maratona de Programação. Nestes experimentos não oficiais, analisava o comportamento e resposta dos alunos sucesso e erros semestre a semestre.
	
Tive a oportunidade de ter contato com a comunidade de maratonistas e pensar antes da parte científica, nas dificuldades dos professores em se aproximares e utilizarem de técnicas que tem dados resultados bem interessantes no meu campus. 

Desde então em 2017 comecei a me envolver de forma isolada, com a construção de uma ferramenta, ainda sem fundamentação cientifica adequada, para auxílio de ensino de programação de computadores. Após 3 versões de testes, em 2020, esta ferramenta existe e está em testes na minha instituição, embasada por artigos científicos ainda simples. 

Durante o ano de 2020, junto com o início do amadurecimento da ferramenta e um posicionamento institucional favorável ao meu afastamento em 2021, me posicionei à ingressar no meu processo de doutoramento com algum tema ligado à esta trajetória.

Dentro das pesquisas iniciais apareceram evidencias de alguns espaços relevantes para a pesquisa dentro da ciência da computação em algumas áreas como Teoria da Computação, IHC e Informática da Educação. Sendo que tenho em minhas mãos uma ferramenta autoral, com potencial de uso e que pode auxiliar de forma relevante os experimentos, o embasamento dos dados, e teste de hipóteses. Apesar de não ter um total embasamento cientifica, ao desenvolver a ferramenta eu sempre tentei criar vários mecanismos de métricas que fossem interessantes para uma possível tese cientifica.

Antes da carreira acadêmica, tive experiências que acho muito relevantes no mercado de trabalho, principalmente os 2 anos como analista de uma equipe no modelo de  \textit{Startup} dentro de uma Empresa de telecomunicações brasileira.

Além disso tenho um trabalho ligado ao empreendedorismo no meu campus, tendo a oportunidade de participar e orientar uma \textit{Startup}\begin{Resumo} de destaque que participou dos programas de aceleração mais importantes de MG como Desafios Sebrae, Lemonade e SEED onde o modelo de negócio passava por coleta e análise de dados em redes socais.

Eis então que apresento meu projeto de tese junto ao DCC da UFMG. 